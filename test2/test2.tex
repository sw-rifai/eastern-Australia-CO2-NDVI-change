%% Copernicus Publications Manuscript Preparation Template for LaTeX Submissions
%% ---------------------------------
%% This template should be used for copernicus.cls
%% The class file and some style files are bundled in the Copernicus Latex Package, which can be downloaded from the different journal webpages.
%% For further assistance please contact Copernicus Publications at: production@copernicus.org
%% https://publications.copernicus.org/for_authors/manuscript_preparation.html

%% copernicus_rticles_template (flag for rticles template detection - do not remove!)

%% Please use the following documentclass and journal abbreviations for discussion papers and final revised papers.

%% 2-column papers and discussion papers
\documentclass[gc, manuscript]{copernicus}



%% Journal abbreviations (please use the same for discussion papers and final revised papers)


% Advances in Geosciences (adgeo)
% Advances in Radio Science (ars)
% Advances in Science and Research (asr)
% Advances in Statistical Climatology, Meteorology and Oceanography (ascmo)
% Annales Geophysicae (angeo)
% Archives Animal Breeding (aab)
% ASTRA Proceedings (ap)
% Atmospheric Chemistry and Physics (acp)
% Atmospheric Measurement Techniques (amt)
% Biogeosciences (bg)
% Climate of the Past (cp)
% DEUQUA Special Publications (deuquasp)
% Drinking Water Engineering and Science (dwes)
% Earth Surface Dynamics (esurf)
% Earth System Dynamics (esd)
% Earth System Science Data (essd)
% E&G Quaternary Science Journal (egqsj)
% European Journal of Mineralogy (ejm)
% Fossil Record (fr)
% Geochronology (gchron)
% Geographica Helvetica (gh)
% Geoscience Communication (gc)
% Geoscientific Instrumentation, Methods and Data Systems (gi)
% Geoscientific Model Development (gmd)
% History of Geo- and Space Sciences (hgss)
% Hydrology and Earth System Sciences (hess)
% Journal of Bone and Joint Infection (jbji)
% Journal of Micropalaeontology (jm)
% Journal of Sensors and Sensor Systems (jsss)
% Magnetic Resonance (mr)
% Mechanical Sciences (ms)
% Natural Hazards and Earth System Sciences (nhess)
% Nonlinear Processes in Geophysics (npg)
% Ocean Science (os)
% Polarforschung - Journal of the German Society for Polar Research (polf)
% Primate Biology (pb)
% Proceedings of the International Association of Hydrological Sciences (piahs)
% Scientific Drilling (sd)
% SOIL (soil)
% Solid Earth (se)
% The Cryosphere (tc)
% Weather and Climate Dynamics (wcd)
% Web Ecology (we)
% Wind Energy Science (wes)

%% Please DO NOT add additional packages or LaTeX commands to the template. They
%% are not supported by Coperncius. LaTeX packages already
%% included in the copernicus.cls are:
%\usepackage[german, english]{babel}
%\usepackage{tabularx}
%\usepackage{cancel}
%\usepackage{multirow}
%\usepackage{supertabular}
%\usepackage{algorithmic}
%\usepackage{algorithm}
%\usepackage{amsthm}
%\usepackage{float}
%\usepackage{subfig}
%\usepackage{rotating}

% Pandoc citation processing

% The "Technical instructions for LaTex" by Copernicus require _not_ to insert any additional packages.
%

\begin{document}

\title{Template for preparing your manuscript submission to Copernicus
journals using RMarkdown}


\Author[1, *]{Daniel}{Nüst}
\Author[2]{Josiah}{Carberry}
\Author[1, *]{Markus}{Konkol}
\Author[3, †]{Nikolaus}{Copernicus}


\affil[1]{Institute for Geoinformatics, University of Münster, 48149
Münster, Germany}
\affil[2]{Psychoceramics, Wesleyan University, Middletown, CT, United
States}
\affil[3]{University of Ferrara, Ferrara, Italy}
\affil[†]{deceased, 24 May 1543}
\affil[*]{These authors contributed equally to this work.}

\runningtitle{R Markdown Template for Copernicus}

\runningauthor{Nüst et al.}


\correspondence{Daniel\ Nüst\ (daniel.nuest@uni-muenster.de)}



\received{}
\pubdiscuss{} %% only important for two-stage journals
\revised{}
\accepted{}
\published{}

%% These dates will be inserted by Copernicus Publications during the typesetting process.


\firstpage{1}

\maketitle


\begin{abstract}
The abstract goes here. It can also be on \emph{multiple lines}.
\end{abstract}


\copyrightstatement{The author's copyright for this publication is
transferred to institution/company.}


\introduction[Introduction]

Introduction text goes here. You can change the name of the section if
necessary using
\texttt{\textbackslash{}introduction{[}modified\ heading{]}}.

The following settings can or must be configured in the header of this
file and are bespoke for Copernicus manuscripts:

\begin{itemize}
\item
  The \texttt{journal} you are submitting to using the official
  abbreviation. You can use the function
  \texttt{rticles::copernicus\_journal\_abbreviations(name\ =\ \textquotesingle{}...\textquotesingle{})}
  to search the existing journals.
\item
  Specific sections of the manuscript:

  \begin{itemize}
  \item
    \texttt{running} with \texttt{title} and \texttt{author}
  \item
    \texttt{competinginterests}
  \item
    \texttt{copyrightstatement} (optional)
  \item
    \texttt{availability} (strongly recommended if any used), one of
    \texttt{code}, \texttt{data}, or \texttt{codedata}
  \item
    \texttt{authorcontribution}
  \item
    \texttt{disclaimer}
  \item
    \texttt{acknowledgements}
  \end{itemize}
\end{itemize}

See the defaults and examples from the skeleton and the official
Copernicus documentation for details.

\textbf{Please note:} Per
\href{https://publications.copernicus.org/for_authors/manuscript_preparation.html}{their
guidelines}, Copernicus does not support additional \LaTeX{} packages or
new \LaTeX{} commands than those defined in their \texttt{.cls} file.
This means that you cannot add any extra dependencies and a warning will
be thrown if so.\\
This extends to syntax highlighting of source code. Therefore this
template sets the parameter \texttt{highlight} in the YAML header to
\texttt{NULL} to deactivate Pandoc syntax highlighter. This prevent
addition of external packages for highlighting inserted by Pandoc.
However, it might be desirable to have syntax highlight available in
preprints or for others reasons. Please see
\texttt{?rmarkdown::pdf\_document} for available options to activate
highlighting.

\textbf{Important}: Always double-check with the official manuscript
preparation guidelines at
\url{https://publications.copernicus.org/for_authors/manuscript_preparation.html},
especially the sections ``Technical instructions for LaTeX'' and
``Manuscript composition''. Please contact Daniel Nüst,
\texttt{daniel.nuest@uni-muenster.de}, with any problems.

\section{Content section one}

\subsection{Subsection Heading Here}

Subsection text here.

\subsubsection{Subsubsection Heading Here}

Subsubsection text here.

\section{Content section with citations}

See the
\href{http://rmarkdown.rstudio.com/authoring_bibliographies_and_citations.html}{R
Markdown docs for bibliographies and citations}.

Copernicus supports biblatex and a sample bibliography is in file
\texttt{sample.bib}. Read \citep{ahlstromDominantRoleSemiarid2015a}, and
\citep[see][]{ainsworthResponsePhotosynthesisStomatal2007}.

\citep{ahlstromDominantRoleSemiarid2015a}

a;dlskjf
\citep{medlynUsingModelsGuide2016c, mooreReviewsSynthesesAustralian2016}

\conclusions[Conclusions]

The conclusion goes here.



\codedataavailability{use this to add a statement when having data sets
and software code
available} %% use this section when having data sets and software code available

\sampleavailability{use this section when having geoscientific samples
available} %% use this section when having geoscientific samples available

\videosupplement{use this section when having video supplements
available} %% use this section when having geoscientific samples available

%%%%%%%%%%%%%%%%%%%%%%%%%%%%%%%%%%%%%%%%%%
%% optional

%%%%%%%%%%%%%%%%%%%%%%%%%%%%%%%%%%%%%%%%%%
\appendix
\section{Figures and tables in appendices}
\subsection{Option 1}

If you sorted all figures and tables into the sections of the text,
please also sort the appendix figures and appendix tables into the
respective appendix sections. They will be correctly named
automatically.

\subsection{Option 2}

If you put all figures after the reference list, please insert appendix
tables and figures after the normal tables and figures.

\texttt{\textbackslash{}appendixfigures} needs to be added in front of
appendix figures \texttt{\textbackslash{}appendixtables} needs to be
added in front of appendix tables

Please add \texttt{\textbackslash{}clearpage} between each table and/or
figure. Further guidelines on figures and tables can be found below.
Regarding figures and tables in appendices, the following two options
are possible depending on your general handling of figures and tables in
the manuscript environment: To rename them correctly to A1, A2, etc.,
please add the following commands in front of them:
\noappendix

%%%%%%%%%%%%%%%%%%%%%%%%%%%%%%%%%%%%%%%%%%
\authorcontribution{Daniel wrote the package. Josiah thought about
poterry. Markus filled in for a second author.} %% optional section

%%%%%%%%%%%%%%%%%%%%%%%%%%%%%%%%%%%%%%%%%%
\competinginterests{The authors declare no competing
interests.} %% this section is mandatory even if you declare that no competing interests are present

%%%%%%%%%%%%%%%%%%%%%%%%%%%%%%%%%%%%%%%%%%
\disclaimer{We like Copernicus.} %% optional section

%%%%%%%%%%%%%%%%%%%%%%%%%%%%%%%%%%%%%%%%%%
\begin{acknowledgements}
Thanks to the rticles contributors!
\end{acknowledgements}

%% REFERENCES
%% DN: pre-configured to BibTeX for rticles

%% The reference list is compiled as follows:
%%
%% \begin{thebibliography}{}
%%
%% \bibitem[AUTHOR(YEAR)]{LABEL1}
%% REFERENCE 1
%%
%% \bibitem[AUTHOR(YEAR)]{LABEL2}
%% REFERENCE 2
%%
%% \end{thebibliography}

%% Since the Copernicus LaTeX package includes the BibTeX style file copernicus.bst,
%% authors experienced with BibTeX only have to include the following two lines:
%%
\bibliographystyle{copernicus}
\bibliography{sample.bib}
%%
%% URLs and DOIs can be entered in your BibTeX file as:
%%
%% URL = {http://www.xyz.org/~jones/idx_g.htm}
%% DOI = {10.5194/xyz}


%% LITERATURE CITATIONS
%%
%% command                        & example result
%% \citet{jones90}|               & Jones et al. (1990)
%% \citep{jones90}|               & (Jones et al., 1990)
%% \citep{jones90,jones93}|       & (Jones et al., 1990, 1993)
%% \citep[p.~32]{jones90}|        & (Jones et al., 1990, p.~32)
%% \citep[e.g.,][]{jones90}|      & (e.g., Jones et al., 1990)
%% \citep[e.g.,][p.~32]{jones90}| & (e.g., Jones et al., 1990, p.~32)
%% \citeauthor{jones90}|          & Jones et al.
%% \citeyear{jones90}|            & 1990

\end{document}
