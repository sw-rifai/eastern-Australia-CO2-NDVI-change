%% Copernicus Publications Manuscript Preparation Template for LaTeX Submissions
%% ---------------------------------
%% This template should be used for copernicus.cls
%% The class file and some style files are bundled in the Copernicus Latex Package, which can be downloaded from the different journal webpages.
%% For further assistance please contact Copernicus Publications at: production@copernicus.org
%% https://publications.copernicus.org/for_authors/manuscript_preparation.html

%% copernicus_rticles_template (flag for rticles template detection - do not remove!)

%% Please use the following documentclass and journal abbreviations for discussion papers and final revised papers.

%% 2-column papers and discussion papers
\documentclass[gc, manuscript]{copernicus}



%% Journal abbreviations (please use the same for discussion papers and final revised papers)


% Advances in Geosciences (adgeo)
% Advances in Radio Science (ars)
% Advances in Science and Research (asr)
% Advances in Statistical Climatology, Meteorology and Oceanography (ascmo)
% Annales Geophysicae (angeo)
% Archives Animal Breeding (aab)
% ASTRA Proceedings (ap)
% Atmospheric Chemistry and Physics (acp)
% Atmospheric Measurement Techniques (amt)
% Biogeosciences (bg)
% Climate of the Past (cp)
% DEUQUA Special Publications (deuquasp)
% Drinking Water Engineering and Science (dwes)
% Earth Surface Dynamics (esurf)
% Earth System Dynamics (esd)
% Earth System Science Data (essd)
% E&G Quaternary Science Journal (egqsj)
% European Journal of Mineralogy (ejm)
% Fossil Record (fr)
% Geochronology (gchron)
% Geographica Helvetica (gh)
% Geoscience Communication (gc)
% Geoscientific Instrumentation, Methods and Data Systems (gi)
% Geoscientific Model Development (gmd)
% History of Geo- and Space Sciences (hgss)
% Hydrology and Earth System Sciences (hess)
% Journal of Bone and Joint Infection (jbji)
% Journal of Micropalaeontology (jm)
% Journal of Sensors and Sensor Systems (jsss)
% Magnetic Resonance (mr)
% Mechanical Sciences (ms)
% Natural Hazards and Earth System Sciences (nhess)
% Nonlinear Processes in Geophysics (npg)
% Ocean Science (os)
% Polarforschung - Journal of the German Society for Polar Research (polf)
% Primate Biology (pb)
% Proceedings of the International Association of Hydrological Sciences (piahs)
% Scientific Drilling (sd)
% SOIL (soil)
% Solid Earth (se)
% The Cryosphere (tc)
% Weather and Climate Dynamics (wcd)
% Web Ecology (we)
% Wind Energy Science (wes)

%% Please DO NOT add additional packages or LaTeX commands to the template. They
%% are not supported by Coperncius. LaTeX packages already
%% included in the copernicus.cls are:
%\usepackage[german, english]{babel}
%\usepackage{tabularx}
%\usepackage{cancel}
%\usepackage{multirow}
%\usepackage{supertabular}
%\usepackage{algorithmic}
%\usepackage{algorithm}
%\usepackage{amsthm}
%\usepackage{float}
%\usepackage{subfig}
%\usepackage{rotating}

% Pandoc citation processing

% The "Technical instructions for LaTex" by Copernicus require _not_ to insert any additional packages.
%

\begin{document}

\title{Thirty-eight years of CO\textsubscript{2} fertilization have
outpaced growing aridity to drive greening of Australian woody
ecosystems}


\Author[1, *]{Sami W.}{Rifai}
\Author[1,2,3,4]{Martin G.}{De Kauwe}
\Author[1,5]{Anna M.}{Ukkola}
\Author[6]{Lucas A.}{Cernusak}
\Author[7,8]{Patrick}{Meir}
\Author[9]{Belinda E.}{Medlyn}
\Author[1,2]{Andy J.}{Pitman}


\affil[1]{ARC Centre of Excellence for Climate Extremes, University of
New South Wales, Sydney, NSW 2052, Australia}
\affil[2]{Climate Change Research Centre, University of New South Wales,
Sydney, NSW 2052, Australia}
\affil[3]{Evolution \& Ecology Research Centre, University of New South
Wales, Sydney, NSW 2052, Australia}
\affil[4]{School of Biological Sciences, University of Bristol, Bristol,
BS8 1TQ, United Kingdom}
\affil[5]{Research School of Earth Sciences, Australian National
University, Canberra, ACT 0200, Australia}
\affil[6]{College of Science and Engineering, James Cook University,
Cairns, QLD 4188, Australia}
\affil[7]{Research School of Biology, The Australian National
University, Acton, ACT 2601, Australia}
\affil[8]{School of Geosciences, University of Edinburgh, Edinburgh
EH89XP, UK}
\affil[9]{Hawkesbury Institute for the Environment, Western Sydney
University, Penrith, NSW 2753, Australia}

\runningtitle{R Markdown Template for Copernicus}

\runningauthor{Nüst et al.}


\correspondence{Sami W.\ Rifai\ (s.rifai@unsw.edu.au)}



\received{}
\pubdiscuss{} %% only important for two-stage journals
\revised{}
\accepted{}
\published{}

%% These dates will be inserted by Copernicus Publications during the typesetting process.


\firstpage{1}

\maketitle


\begin{abstract}
Climate change is projected to increase the imbalance between the supply
(precipitation) and atmospheric demand for water (i.e.~increased
potential evapotranspiration), stressing plants in water-limited
environments. Plants may be able to offset increasing aridity because
rising CO\textsubscript{2} increases water-use-efficiency.
CO\textsubscript{2} fertilization has also been cited as one of the
drivers of the widespread `greening' phenomenon. However, attributing
the size of this CO\textsubscript{2} fertilization effect is
complicated, due in part to a lack of long-term vegetation monitoring
and interannual to decadal-scale climate variability. In this study we
asked the question, how much has CO\textsubscript{2} contributed towards
greening? We focused our analysis on a broad aridity gradient spanning
eastern Australia's woody ecosystems. Next we analysed 38-years of
satellite remote sensing estimates of vegetation greenness (normalized
difference vegetation index, NDVI) to examine the role of
CO\textsubscript{2} in ameliorating climate change impacts. Multiple
statistical techniques were applied to separate the
CO\textsubscript{2}-attributable effects on greening from the changes in
water supply and atmospheric aridity. Widespread vegetation greening
occurred despite a warming climate, increases in vapor pressure deficit,
and repeated record-breaking droughts and heatwaves. Between 1982-2019
we found that NDVI increased (median 11.3\%) across 90.5\% of the woody
regions. After masking disturbance effects (e.g.~fire), we statistically
estimated an 11.7\% increase in NDVI attributable to
CO\textsubscript{2}, broadly consistent with a hypothesized theoretical
expectation of an 8.6\% increase in water-use-efficiency due to rising
CO\textsubscript{2}. In contrast to reports of a weakening
CO\textsubscript{2} fertilization effect, we found no consistent
temporal change in the CO\textsubscript{2} effect. We conclude rising
CO\textsubscript{2} has mitigated the effects of increasing aridity,
repeated record-breaking droughts, and record-breaking heat waves in
eastern Australia. However, we were unable to determine whether trees or
grasses were the primary beneficiary of the CO\textsubscript{2} induced
change in water-use-efficiency, which has implications for projecting
future ecosystem resilience. A more complete understanding of how
CO\textsubscript{2} induced changes in water-use-efficiency affect trees
and non-tree vegetation is needed.
\end{abstract}


\copyrightstatement{The author's copyright for this publication is
transferred to institution/company.}


\introduction[Introduction]Australia is the world's driest inhabited
continent. Predicting how climate change will affect ecosystem
resilience and alter Australia's terrestrial hydrological cycle is of
paramount importance. Australia's woody ecosystems are mostly
concentrated in the east, where there are large gradients of
precipitation (P) (300 - \textgreater2000 mm yr-1) and potential
evapotranspiration (PET) (800-2100 mm yr-1). Most eastern Australian
woodlands occupy water-limited regions where annual PET far exceeds P
(Fig. S1), and tree species have evolved to cope with water-limited
conditions (Peters et al.~2021) and high interannual rainfall
variability. However, the climate is warming: eight of Australia's ten
warmest years on record have occurred since 2005 (CSIRO \& Bureau of
Meteorology, 2020) and Australia's climate has warmed by
\textasciitilde{} 1.5℃ since records began in 1910. The warming has
likely increased atmospheric demand for water (e.g.~PET or vapor
pressure deficit, VPD). In most woody ecosystems, the ratio of water
supply (i.e.~P) to water demand (i.e.~PET) has declined in recent
decades (Figs. 1,2a). Eastern Australia has also been impacted by
several multi-year droughts, episodic deluges of rainfall
\citep{king_role_2020}, and an increasing frequency of severe heatwaves
(Perkins et al., 2012) in the last few decades. Precipitation changes
have been spatially variable over eastern Australia, where northern
Queensland grew wetter and southeast Australia grew drier (Fig. S2). In
the last two decades, southeast Australia experienced the two worst
droughts in the observational record (2001-2009; van Dijk et al., 2013
and 2017-2019; Bureau of Meteorology). Yet between these two droughts,
Eastern Australia experienced record breaking rainfall in 2011
associated with a strong La Niña event. This caused marked vegetation
`greening' (e.g.~increased foliar cover), even in the arid interior
(Bastos et al., 2013; Poulter et al., 2014; Ahlstrom et al., 2015).
However, this greening contributed to record-breaking fires in the
following year (Harris et al., 2018).

Theory suggests that plant physiological responses to atmospheric carbon
dioxide (CO2) may mitigate some of the negative effects of an aridifying
climate. However, the magnitude of plant responses to increased
atmospheric CO2 has been challenging to establish in field experiments
(Jiang et al., 2020b), from observations (Zhu et al., 2016; Walker et
al., 2020), or to separate from other drivers (e.g.~climate variability,
disturbances, and changes in land management). Studies have used data
from the Advanced Very High Resolution Radiometer (AVHRR) satellites to
show positive trends in the normalized difference vegetation index
(NDVI) over Australia (Donohue et al., 2009). The greening trend is
caused by increased leaf area, which has resulted from increased
atmospheric CO2 concentrations (Donohue et al., 2013; Ukkola et al.,
2016). The evidence for increases in leaf area from rising CO2 have also
been supported by observations of reduced runoff in Australia's drainage
basins (Trancoso et al., 2017; Ukkola et al., 2016).

Yet disentangling the CO2 fertilization effect from other drivers of
climate variability and global change has been particularly challenging
for satellite based analyses. It is challenging to attribute causes of
greening because of co-occurring changes in climate, land-use, and
disturbance are confounded with the effect of CO2 fertilization.
Furthermore, the time series of even the longest systematically
collected optical vegetation index records from a single sensor is 20
years (e.g.~MODIS Terra). Analysis of trends extending beyond 20 years
requires merging satellite records across sensors and platforms. But
this requires care to address changes in radiometric and spatial
resolution of the sensor, as well as drift in the solar zenith angle (Ji
\& Brown, 2017; Frankenberg et al., 2021) and the time of retrieval.
Thus different analytical methodologies have produced disagreements over
where greening has occurred (Cortés et al., 2021). One often-used method
to provide additional constraint on greening trends has been to compare
remote sensing derived trends with modeled changes in leaf area index
(LAI) from ensembles of dynamic global models (Zhu et al., 2016; Wang et
al., 2020). However these model attribution approaches rely on a set of
key assumptions. None of the models can accurately predict LAI changes
in response to rising CO2 (De Kauwe et al., 2014; Medlyn et al., 2016).
Vegetation models have been shown to diverge in their simulation of LAI
over Australia (Medlyn et al., 2016; Zhu et al., 2016; Teckentrup et
al., in review), and have bioclimatic rules for determining phenology
which may not be appropriate for the highly variable Australian climate
and the evergreen Eucalyptus forests (Teckentrup et al., 2021). These
model simulations are typically compared with modeled LAI products
derived from the red and near infrared wavelengths of multispectral
satellite sensors, of which each product carries specific algorithmic
assumptions about canopy-light interception which are conditional upon
estimated land cover types. In comparison, NDVI carries no ecosystem
specific assumption, and is an effective proxy of leaf area in
ecosystems with low-to-moderate canopy cover (Carlson \& Ripley, 1997),
a characteristic of eastern Australian woody ecosystems (Specht, 1972;
Yang et al., 2018).

Here we ask, how much can greening trends be explained by rising CO2?
Using eastern Australia as a model system, we used a multi satellite
derived NDVI record encompassing 38 years to isolate the influence of
CO2 from simultaneous effects of meteorological change and disturbance.
Next we contrasted CO2 effects with theoretical predictions based on
water-use-efficiency (WUE) theory for plants and the observed rise in
CO2. Finally, we examined whether recent NDVI greening trends have
co-occurred with changes in tree or grass cover over the last two
decades.

\section{Methods}

\subsection{Study area}

The study region encompasses the dominant woody ecosystems of eastern
Australia (Fig. S1b). We used the National Vegetation Information System
5.1 land cover dataset (Table S1) to select locations designated as
``Acacia Forests and Woodlands'', ``Acacia Open Woodlands'', ``Callitris
Forests and Woodlands'', ``Casuarina Forests and Woodlands'', ``Eucalypt
Low Open Forests'', ``Eucalypt Open Forests'', ``Eucalypt Open
Woodlands'', ``Eucalypt Tall Open Forests'', ``Eucalypt Woodlands'',
``Low Closed Forests and Tall Closed Shrublands'', ``Mallee Open
Woodlands and Sparse Mallee Shrublands'', ``Mallee Woodlands and
Shrublands'', ``Melaleuca Forests and Woodlands'', ``Other Forests and
Woodlands'', ``Other Open Woodlands'', ``Rainforests and Vine
Thickets'', and ``Tropical Eucalypt Woodlands/Grasslands''.

\subsection{Climate and remote sensing datasets}

We used the atmospheric CO2 record from the deseasonalized Mauna Loa
record (https://www.esrl.noaa.gov/gmd/ccgg/trends/data.html), and
extracted climate data (Table S1) from the Australian Bureau of
Meteorology's Australian Water Availability Project (AWAP; Jones, et
al.~2009). AWAP is a gridded climate product interpolated to 0.05° from
a large network of meteorological stations distributed across Australia.
Vapor pressure deficit was calculated using daily estimates of maximum
temperature and vapor pressure at 15:00 hours. PET was calculated from
shortwave radiation and mean air temperature using the Priestley-Taylor
method (Davis et al., 2017). The Priestley-Taylor method has been shown
to be appropriate for estimating large-scale PET (Raupach, 2000) and is
more suited for use in long-term analysis where CO2 increased than other
common formulations such as the Penman-Monteith equation (Milly \&
Dunne, 2016; Greve et al., 2019), which explicitly imposes a fixed
stomatal resistance that is incompatible with plant physiology theory
(Medlyn et al., 2001). AWAP measurements of shortwave radiation only
extend back to 1990, so we extended the PET record to 1982 by
calibrating the ERA5-Land PET record (1980-2019) to the AWAP PET record
(1990-2019) by linear regression for each grid-cell, and then gap-filled
the years 1982-1989 with the calibrated ERA5 PET. PET from the Climate
Research Unit record (Harris et al., 2014) was highly correlated with
both the recalibrated ERA5 PET (r = 0.91; 1982-1989) and the original
AWAP PET (r = 0.97; 1990-2019). Next, we calculated a 30-year
climatology of the meteorological variables using the period of
1982-2011 to be close to current standards (World Meteorological
Organization, 2017). We used this climatology to define the mean annual
P:PET (MIMA), and as the reference to calculate a 12-month running
anomaly of annual P:PET (MIanom). Zonal statistics for each
meteorological variable were calculated using simplified Köppen climate
zones, derived from the Australian Bureau of Meteorology (Fig. 2b, Table
S1).

We used surface reflectance from two satellite products to generate the
NDVI record: National Oceanic and Atmospheric Administration's Climate
Data Record v5 Advanced Very High Radiometric Resolution (AVHRR) Surface
Reflectance (NOAA-CDR) record and the National Space and Aeronautical
Administration's MCD43A4 Nadir Bidirectional Reflectance Distribution
Function Adjusted Reflectance (MODIS-MCD43) (Table S1; Schaaf \& Wang,
2015). NDVI data was extracted from 1982-2019 at 0.05° resolution from
the NOAA-CDR AVHRR version 5 product (Vermote \& NOAA CDR Program,
2018). The surface reflectance record of AVHRR extends through 2019, but
the quality of the record starts to degrade in 2017 because of an
increase in the solar zenith angle (Ji \& Brown, 2017), causing a
sensor-produced decline in NDVI during 2017-2019. For this reason we
only use AVHRR surface reflectance data between 1982-2016. We composited
monthly mean AVHRR NDVI (NDVIAVHRR) estimates using only daily pixel
retrievals with no detected cloud cover (Quality Assurance band, bit 1).
Monthly NDVIAVHRR estimates aggregated from less than three daily
retrievals were removed. They were also removed when the coefficient of
variation of daily retrievals for a given month was greater than 25\%.
We also removed NDVIAVHRR monthly estimates where NDVIAVHRR, solar
zenith angle, or time of acquisition deviated beyond 3.5 standard
deviations from the monthly mean, calculated from a climatology spanning
1982-2016.

We used the MODIS-MCD43 surface reflectance at 500 m resolution to
derive NDVI for 2001-2019 (NDVIMODIS). Monthly mean estimates of the
surface reflectance were produced by compositing pixels flagged as
``ideal-quality'' (Quality Assurance, bits 0-1). We also masked
disturbances to have greater confidence in our attribution of the
targeted drivers of NDVIMODIS change (climate \& CO2). The Global Forest
Change product v1.7 (Hansen et al., 2013) was used to mask pixels from
2001 onwards that had experienced forest loss due to deforestation or
severe stand clearing disturbance. We masked pixel locations that
experienced bushfires from the year 2001 onwards. Specifically, these
pixels were masked for the year of burning and the following three years
using the 500 m resolution MODIS-MCD64 monthly burned-area product
(Giglio et al., 2018). We terminated the NDVIMODIS time series in August
of 2019, prior to the widespread bushfires of late 2019/2020. Both
NDVIAVHRR and NDVIMODIS datasets were processed using Google Earth
Engine (Gorelick et al., 2017), and exported at 5 km spatial resolution,
which best approximated the native resolution of the NOAA-CDR AVHRR and
AWAP products. Further post-processing used the `stars' (Pebesma, 2020)
and `data.table' (Dowle \& Srinivasan, 2019) R packages (see code
availability section).

We merged the processed 1982-2016 NDVI\textsubscript{AVHRR} with the
2001-2019 NDVIMODIS by recalibrating the NDVIAVHRR with a generalized
additive model (GAM). Specifically, we used one million observations
from the overlapping 2001-2016 portion of both records to fit a GAM
using the `mgcv' R package (Wood, 2017) to model NDVIMCD64 from AVHRR
derived covariates as: \begin{equation}
NDVI_{MOD} = s(NDVI_{AVHRR})+s(month) + s(SZA) + s(TOD) + s(x,y)
\end{equation}

where `s' represents a penalized smoothing function using a thin-plate
regression spline, SZA is the solar zenith angle, NDVIAVHRR is the
uncalibrated NDVI from AVHRR, TOD is time of day of retrieval, and x and
y represent longitude and latitude, respectively. The fit GAM was then
used to generate the recalibrated AVHRR NDVI. The merged NDVI dataset
was created by joining the 1982-2000 recalibrated AVHRR NDVI with the
2001-2019 NDVIMODIS. We further reduced monthly temporal variability of
NDVI by calculating a three month rolling mean of NDVI which we used for
subsequent statistical model fitting.

\#\#Estimating NDVI and climate trends We estimated the relative
increase in NDVI between 1982-2019 with respect to time (equation 2) for
each grid cell with an iteratively weighted least squares robust linear
model via the `rlm' function in R's MASS package (Venables \& Ripley,
2002) as follows. \begin{equation}
NDVI=\beta_0+ \beta_1\,year+\beta_2\,sensor
\end{equation}

Here \(\beta_0\) represents the estimated NDVI in 1982, the year term
starts at 1982, and the sensor term is a binary covariate that accounts
for residual offset differences between the recalibrated AVHRR NDVI ,
and the NDVIMOD. The relative temporal trends for climate variables and
the MODIS vegetation continuous fractions were fit for each grid cell
location using the Theil-Sen estimator, a form of robust pairwise
regression, with the `zyp' R package (Bronaugh \& Consortium, 2019). The
temporal covariate was recentered to start with the first hydrological
year (where the year starts one month earlier in December) of the data
so that the intercept term represents the mean at the start of the time
series. The relative rate of change for each variable was reconstructed
by calculating \begin{equation}
100*[\frac{\beta_1(year_{end}-year_{start})}{\beta_0}]
\end{equation}

where \(\beta_0\) and \(\beta_1\) are the intercept and trend derived
from Theil-Sen regression.

\subsection{Estimating contribution of CO2 and climate toward NDVI
trends}

We used the merged NDVI observations to fit multiple statistical models
to quantify the impact of changes in CO2 and meteorological variables on
NDVI. The relationship between NDVI and the running 12-month mean of
P:PET was strongly nonlinear and followed a monotonic saturating
sigmoidal relationship as indicated by GAM fits (methods equation 6, see
below). GAMs can characterize a nonlinear response without specifying a
functional form, yet the underlying spline parameters are not easily
interpreted as the parameters of a fixed nonlinear function. Therefore
we used nonlinear least squares (nls\_multstart function (Padfield \&
Matheson, 2020) in R v4.01) to compare model fits to a set of fixed
nonlinear functional forms including the Weibull function (equation 4;
Fig. 4), the logistic function (equation 5; Fig. S5), and the Richards
growth function (equation 6; Fig. S6). The focus on the Weibull models
because they showed equivalent goodness of fit with fewer parameters
than the Richards function models. Next we added a linear modifier to
the Weibull function using the covariates of CO2 (ppm) and the ratio of
the anomaly of P:PET (MIanom) to the mean annual P:PET (MIMA) as
follows: \begin{align}
NDVI=V_a-V_d[exp(-exp(c_{ln})\,(MI_{MA})^{q})]+\eta\\
\eta = \beta_{1}\frac{MI_{anom}}{MI_{MA}}+\beta_{2}\,CO_2\,MI_{MA} +\beta_{3}\,CO_2\,\frac{MI_{anom}}{MI_{MA}}+sensor\nonumber
\end{align}

Here the sensor term is a binary covariate indicating the AVHRR or MODIS
platform. Model-fitted parameters Va and Vd correspond to the asymptote,
and the asymptote's difference from the minimum NDVI, while cln is the
logarithm of the rate constant, and q is the power to which MIMA is
raised. The model was fit by individual season with one million
observations per model fit. Corresponding goodness-of fit metrics were
calculated by season (R2 and root mean square error; Fig. 5) with one
million randomly sampled observations. Alternative nonlinear functional
forms were also fit to characterize the effect of CO2 upon NDVI. A
logistic model was fit across space for each hydrological year as

\begin{equation}
NDVI = \frac{V_A}{(1+exp((m-MI_{12mo})/s))}
\end{equation}

where NDVI is the hydrological year mean value of NDVI for a grid cell
location, m is the midpoint, s is a scale parameter, and VA is the
asymptote (plotted in Fig. S5). We also used a modified Richards growth
function to characterize the CO2 effect upon seasonal NDVI (Fig. S6) as

\begin{align}
NDVI=(V_A+\beta_1\,CO_2+\beta_2\,MI_{f.anom})\,\frac{(1+exp(m+\beta_3\,CO_2+\beta_4\,MI_{f.anom} - MI_{MA}))}{(s+\beta_5\,CO_2+\beta_6\,MI_{f.anom})^{(-exp(-(q+\beta_7\,CO_2+\beta_8\,F)))}}\\
MI_{f.anom} = \frac{MI_{anom}}{MI_{MA}}\nonumber
\end{align}

Here the \(\beta\) terms act to linearly modify the core nonlinear
parameters (VA, m, s, q) with the effects of CO2 and MIFanom. Each
seasonal model component was fit across space with one million random
samples from the total merged NDVI record (approx 14.3 million
observations).

To ensure consistent interpretation of the nonlinear response across
P:PET, we also fit linear models explaining NDVI with CO2 and MIanom by
season in MIMA bin-widths of 0.2 (equation 7; Fig. S4). Separate linear
models were fit for increments of 0.15 of MIMA for each season using the
merged 1982-2019 NDVI record. NDVI was modeled as

\begin{equation}
NDVI = \beta_0+\beta_1\,CO_2 + \beta_2\,MI_{anom}+\beta_3\,Veg.\,Class+\beta_4\,sensor
\end{equation}

where MIanom is the annual anomaly of P:PET, Veg. Class is the NVIS 5.1
vegetation class, and sensor is a binary variable used to account for
residual differences between the recalibrated AVHRR NDVI and NDVIMOD
records. To aid the comparison of model effects, we centered and
standardized the continuous model before regression. The standardized
CO2 and P:PETanom. effects (\(\beta\)) are presented in Fig. S4.

Next, we fit robust multiple linear regression models to the time series
of NDVI for each of the 39,463 pixel locations. The CO2 effect for each
grid cell location was simultaneously estimated with the linear effects
of the anomalies (anom) of P, PET, VPD, and MI as fractions of their
mean annual values (MA) as follows.

\begin{equation}
NDVI=\beta_0+ \beta_1\,CO_2+\beta_2\,\frac{P_{anom}}{P_{MA}}+\beta_3\,\frac{PET_{anom}}{PET_{MA}}+\beta_4\,\frac{VPD_{anom}}{{VPD_{MA}}}+\beta_5\,sensor
\end{equation}

Finally we estimated the CO2 effect across the study region using a GAM
with a penalized smoothing function (s) characterizing the effect of the
anomalies and mean annual values of VPD, P, and PET and sensor epoch as
follows.

\begin{equation}
NDVI = s(MI_{MA},CO_2) + s(VPD_{anom},VPD_{MA})+s(P_{anom},P_{MA})+s(PET_{anom},PET_{MA})+sensor
\end{equation}

\subsection{A simplified theoretical water use efficiency model}

We compared the statistically attributed CO2 amplification of NDVI with
the expectation from a simple theoretical model of WUE. Following
Donohue et al.~(2013), WUE (W) is defined as: \begin{equation}
W_{leaf} = \frac{A_{leaf}}{E_{leaf}} = \frac{C_a}{1.6D}(1 - \chi)
\end{equation}

where A is leaf level assimilation \(umol\,m^{2}\,s^{-1}\), E is leaf
level transpiration \(mol\,m^{2}\,s^{-1}\), Ca is atmospheric CO2
\(umol\,umol^{-1}\), Ci is intercellular CO2 \(umol\,umol^{-1}\),
\(\chi\) is Ci/Ca, and D is atmospheric vapor pressure deficit
\(mol\,mol^{-1}\). The relative rate of change in W with respect to a
change in Ca can be calculated as:

\begin{equation}
\frac{dW_{leaf}}{W_{leaf}}=\frac{dA_{leaf}}{A_{leaf}} - \frac{dE_{leaf}}{E_{leaf}} = \frac{dC_a}{C_a} - \frac{dD}{D} + \frac{d(1-\chi)}{(1-\chi)}
\end{equation}

If temperature increases without a corresponding increase in humidity, D
increases which also causes transpiration to rise and thus reduces W.
However, W is predicted to increase with CO2 which may offset increases
in D. Experiments suggest that \(\chi\) does not change with Ca but is
sensitive to D (Wong et al., 1985; Drake et al., 1997) and can be
estimated as being proportional to the square root of D (Medlyn et al.,
2011). By substituting \[(1-\chi) \approx \sqrt(D)\] into equation (11)
we can estimate the theoretical combined effect of Ca and D upon Wleaf
as: \begin{equation}
\frac{dW_{leaf}}{W_{leaf}}=\frac{dA_{leaf}}{A_{leaf}} - \frac{dE_{leaf}}{E_{leaf}} = \frac{dC_a}{C_a} - \frac{1}{2}\frac{dD}{D}
\end{equation}

Transpiration per unit ground area is strongly controlled by water
supply in warm, water limited environments with relatively low leaf area
such as eastern Australia (Specht, 1972) therefore we approximate canopy
transpiration (Ecanopy) as: \begin{equation}
E_{canopy}=E_{leaf}\,L
\end{equation}

The change in Ecanopy can then be defined as: \begin{equation}
\frac{dE_{canopy}}{E_{canopy}} \approx \frac{dE_{leaf}}{E_{leaf}}+\frac{dL}{L}
\end{equation}

If we assume there is no long-term overall change in precipitation then
we can assume change in Ecanopy is tightly coupled to the water supply,
therefore we have: \begin{equation}
-\frac{dE_{leaf}}{E_{leaf}} \approx \frac{dL}{L}
\end{equation}

NDVI is linearly related to foliar cover (F) until LAI \(\approx\) 3
\(m^2\,m^{-2}\) (Carlson \& Ripley, 1997), which is the predominantly
the case when P:PET \textless{} 1. Most woody ecosystems of eastern
Australia are strongly water limited with LAI \textless= 1
\(m^2\,m^{-2}\), where NDVI is approximately proportional with the
fraction of foliar cover: \begin{equation}
\frac{dL}{L}\approx\frac{dF}{F}\approx\frac{d NDVI}{NDVI}
\end{equation}

Then substituting equation (15) into equation (12) gives:
\begin{equation}
\frac{dW_{leaf}}{W_{leaf}} \approx  \frac{dA_{leaf}}{A_{leaf}} + \frac{dF}{F} \approx \frac{dC_a}{C_a} - \frac{1}{2}\frac{dD}{D}
\end{equation}

If we assume that the benefit towards Wleaf from rising Ca is split
evenly between the relative changes in Aleaf and F, we can predict the
change towards NDVI to be \begin{equation}
\frac{dNDVI}{NDVI} \approx \frac{1}{2}[\frac{dCa}{Ca}-\frac{dD}{2\,D}]
\end{equation}

We compared the WUE theoretical model with the robust linear models fit
for each pixel location (equation 8), and the GAM (equation 9) fit
across the study region. The WUE theoretical model assumes no change in
P, but does account for changes in VPD. Therefore in using the
statistical models to compare with the WUE predictions, we generated
counterfactual predictions from the statistical models with no
precipitation anomaly but with the observed increases in CO2 and VPD.
One weakness with the application of this WUE theoretical model is the
uncertainty regarding the assumed allocation of the Wleaf benefit
towards either Aleaf or F (e.g.~LAI; see above). Donohue et al., (2017)
proposed a similar model to eq (18), the Partitioning of Equilibrium
Transpiration and Assimilation (PETA) hypothesis where the relative
allocation to leaf area is predicted to decline with increasing resource
availability (which could be inferred from growing season LAI). We
calculated the expectation from the PETA hypothesis as another point of
comparison with the CO2 attributable effect on NDVI.

\conclusions[Conclusions]



\codedataavailability{use this to add a statement when having data sets
and software code
available} %% use this section when having data sets and software code available

\sampleavailability{use this section when having geoscientific samples
available} %% use this section when having geoscientific samples available

\videosupplement{use this section when having video supplements
available} %% use this section when having geoscientific samples available

%%%%%%%%%%%%%%%%%%%%%%%%%%%%%%%%%%%%%%%%%%
%% optional

%%%%%%%%%%%%%%%%%%%%%%%%%%%%%%%%%%%%%%%%%%
\appendix
\section{Figures and tables in appendices}
\subsection{Option 1}

If you sorted all figures and tables into the sections of the text,
please also sort the appendix figures and appendix tables into the
respective appendix sections. They will be correctly named
automatically.

\subsection{Option 2}

If you put all figures after the reference list, please insert appendix
tables and figures after the normal tables and figures.

\texttt{\textbackslash{}appendixfigures} needs to be added in front of
appendix figures \texttt{\textbackslash{}appendixtables} needs to be
added in front of appendix tables

Please add \texttt{\textbackslash{}clearpage} between each table and/or
figure. Further guidelines on figures and tables can be found below.
Regarding figures and tables in appendices, the following two options
are possible depending on your general handling of figures and tables in
the manuscript environment: To rename them correctly to A1, A2, etc.,
please add the following commands in front of them:
\noappendix

%%%%%%%%%%%%%%%%%%%%%%%%%%%%%%%%%%%%%%%%%%
\authorcontribution{Daniel wrote the package. Josiah thought about
poterry. Markus filled in for a second author.} %% optional section

%%%%%%%%%%%%%%%%%%%%%%%%%%%%%%%%%%%%%%%%%%
\competinginterests{The authors declare no competing
interests.} %% this section is mandatory even if you declare that no competing interests are present

%%%%%%%%%%%%%%%%%%%%%%%%%%%%%%%%%%%%%%%%%%
\disclaimer{We like Copernicus.} %% optional section

%%%%%%%%%%%%%%%%%%%%%%%%%%%%%%%%%%%%%%%%%%
\begin{acknowledgements}
Thanks to the rticles contributors!
\end{acknowledgements}

%% REFERENCES
%% DN: pre-configured to BibTeX for rticles

%% The reference list is compiled as follows:
%%
%% \begin{thebibliography}{}
%%
%% \bibitem[AUTHOR(YEAR)]{LABEL1}
%% REFERENCE 1
%%
%% \bibitem[AUTHOR(YEAR)]{LABEL2}
%% REFERENCE 2
%%
%% \end{thebibliography}

%% Since the Copernicus LaTeX package includes the BibTeX style file copernicus.bst,
%% authors experienced with BibTeX only have to include the following two lines:
%%
\bibliographystyle{copernicus}
\bibliography{Oz\_PPET\_CO2.bib}
%%
%% URLs and DOIs can be entered in your BibTeX file as:
%%
%% URL = {http://www.xyz.org/~jones/idx_g.htm}
%% DOI = {10.5194/xyz}


%% LITERATURE CITATIONS
%%
%% command                        & example result
%% \citet{jones90}|               & Jones et al. (1990)
%% \citep{jones90}|               & (Jones et al., 1990)
%% \citep{jones90,jones93}|       & (Jones et al., 1990, 1993)
%% \citep[p.~32]{jones90}|        & (Jones et al., 1990, p.~32)
%% \citep[e.g.,][]{jones90}|      & (e.g., Jones et al., 1990)
%% \citep[e.g.,][p.~32]{jones90}| & (e.g., Jones et al., 1990, p.~32)
%% \citeauthor{jones90}|          & Jones et al.
%% \citeyear{jones90}|            & 1990

\end{document}
